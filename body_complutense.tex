\chapter{Introduction}\label{introduction}

Naylang is an open source REPL interpreter, runtime and debugger for the
Grace programming language written in modern C++; It currently
implements a subset of Grace described later, but as both the language
and the interpreter evolves the project strives for
feature-completeness.

\section{Motivation}\label{motivation}

Grace is a language aimed to help novice programmers get acquainted with
the process of programming, and as such it provides safety and
flexibility in it's design.

However, this flexibility comes at a cost, and most of the current
implementations of Grace are in themselves opaque and obscure. Since
Grace is open source, most of it's implementations are also open source,
but this lack of clarity in the implementation makes them hard to extend
and modify by third parties and newcomers, severely damaging the growth
opportunities of the language.

\section{Objectives and Methodology}\label{objectives-and-methodology}

Naylang strives to be an exercise in interpreter construction not only
for the creators, but also for any possible contributor. Therefore, the
project focuses on the following goals:

\begin{itemize}
\tightlist
\item
  To provide a solid implementation of a relevant subset of the Grace
  language.
\item
  To be as approachable as possible by both end users, namely first-time
  programmers, and project collaborators.
\item
  To be itself a teaching tool to learn about one possible
  implementation of a language as flexible as Grace.
\end{itemize}

To that end, the project follows a Test Driven Development approach
\cite{needed}, in which unit tests are written in parallel to or before
the code, in very short iterations. This is the best approach for two
reasons:

Firstly, it provides an easy way to verify which part of the code is
working at all times, since tests strive for complete code coverage
\cite{needed}. Therefore, newcomers to the project will know where
exactly their changes affect the software as a whole, which will allow
them to make changes with more confidence.

Secondly, the tests themselves provide documentation that is always
up-to-date and synchronized with the code. This, coupled with
descriptive test names, provide a myriad of \textbf{working code
examples}. Needless to say that this would result in vital insight
gained at a much quicker pace by a student wanting to learn about
interpreters.

\section{Tradeoffs}\label{tradeoffs}

Since Naylang is designed as a learning exercise, clarity of code and
good software engineering practices will take precedence over
performance in almost every case. More precisely, if there is a simple
and robust yet naïve implementation of a part of the system, that will
be selected instead of the more efficient one.

However, good software engineering practices demand that the
architecture of the software has to be modular and loosely coupled.
This, in addition to the test coverage mentioned earlier, will make the
system extensible enough for anyone interested to modify the project to
add a more efficient implementation of any of it's parts.

In short, the project optimizes for \textbf{approachability} and
\textbf{extensibility}, not for \textbf{execution time} or
\textbf{memory usage}.

\chapter{State of the art}\label{state-of-the-art}

\section{Minigrace}\label{minigrace}

\section{Kernan}\label{kernan}

\section{TreeGraph}\label{treegraph}

\section{JavaScript interpreters}\label{javascript-interpreters}

JavaScript is similar to a fully dynamically typed version of Grace (our
subset).

\chapter{Implementation}\label{implementation}

\section{Abstract Syntax Tree}\label{abstract-syntax-tree}

\section{Object and Execution Model}\label{object-and-execution-model}

\section{Built-in methods and
Prelude}\label{built-in-methods-and-prelude}

\section{Heap and Garbage Collection}\label{heap-and-garbage-collection}

\section{Debugger}\label{debugger}

\section{Frontend}\label{frontend}

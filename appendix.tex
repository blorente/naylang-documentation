\chapter{Appendix A: Title of appendix
A}\label{appendix-a-title-of-appendix-a}

\section{Subtitle 1}\label{subtitle-1}

\section{Subtitle 2}\label{subtitle-2}

\chapter{Appendix B: How was this document
made?}\label{appendix-b-how-was-this-document-made}

\section{Author}\label{author}

\textbf{Note:} the process described in this Appendix was devised by
Álvaro Bermejo, who published it under the MIT license in 2017
{[}@persimmon{]}.

\section{Process}\label{process}

This document was written on Markdown, and converted to PDF using
Pandoc.

Document is written on Pandoc’s extended Markdown, and can be broken
amongst different files. Images are inserted with regular Markdown
syntax for images. A YAML file with metadata information is passed to
pandoc, containing things such as Author, Title, font, etc\ldots{} The
use of this information depends on what output we are creating and the
template/reference we are using.

\section{Diagrams}\label{diagrams}

Diagrams are were created with LaTeX packages such as tikz or pgfgantt,
they can be inserted directly as PDF, but if we desire to output to
formats other than LaTeX is more convenient to convert them to .png
files with tools such as \texttt{pdftoppm}.

\section{References}\label{references}

References are handled by pandoc-citeproc, we can write our bibliography
in a myriad of different formats: bibTeX, bibLaTeX, JSON, YAML,
etc\ldots{}, then we reference in our markdown, and that reference works
for multiple formats
